% \documentclass[manuscript]{geophysics}
\documentclass[paper,twocolumn,twoside]{geophysics}
% \documentclass[paper]{geophysics}
% \documentclass[paper, revised]{geophysics}
% \documentclass[manuscript,revised]{geophysics}
%%fakesection ===    PACKAGES & DEFINITIONS    ===
% \usepackage{showframe}

%% ~ ADDITIONAL PACKAGES TO GEOPHYSICS.CLS
\DeclareGraphicsExtensions{.pdf,.png,.jpg}
\usepackage{color}
\usepackage{tabularx}
\usepackage{colortbl}
\usepackage{booktabs}
\usepackage[USenglish]{babel}
\usepackage[utf8]{inputenc}
\usepackage{lmodern}
\usepackage[T1]{fontenc}
\usepackage{amssymb, amsmath, amsfonts}
\usepackage{upquote}
\usepackage[pdftex, final]{hyperref}
\hypersetup{allcolors=blue, allbordercolors={0 0 .5}, colorlinks=true}

\usepackage{listings}

% Figure directory
\renewcommand{\figdir}{./figures}
\usepackage{xspace}
\newcommand{\empymod}{\texttt{empymod}\xspace}
\usepackage[strings]{underscore}

\definecolor{MyGray}{gray}{0.85}
\newcolumntype{Y}{>{\columncolor{MyGray}\raggedleft\arraybackslash}r}
\newcolumntype{W}{>{\raggedleft\arraybackslash}r}
\newcolumntype{A}{>{\columncolor{MyGray}\raggedleft\arraybackslash}X}
\newcolumntype{B}{>{\raggedleft\arraybackslash}X}

\hyphenation{iso-tro-pic pe-ne-tra-ting}

\newcommand{\rmk}[1]{\textbf{\color{red}{#1}}}
\newcommand{\mr}[1]{\mathrm{#1}}
\frenchspacing

\begin{document}

\title{A digital filter designing tool for the Hankel and Fourier transforms in
potential-, diffusive-, and wavefield modeling}

\renewcommand{\thefootnote}{\fnsymbol{footnote}}

\ms{GEO-2018-????}

\address{
\footnotemark[1]Instituto Mexicano del Petróleo,
Eje Central Lázaro Cárdenas Norte 152,
Col. San Bartolo Atepehuacan C.P. 07730,
Ciudad de México, México,
E-mail: \href{mailto:dieter@werthmuller.org}{Dieter@Werthmuller.org};
\footnotemark[2]Lamont-Doherty Earth Observatory,
305C Oceanography,
61 Route 9W, PO Box 1000,
Palisades NY 10964-8000 US,
E-mail: \href{mailto:KKey@ldeo.columbia.edu}{KKey@ldeo.columbia.edu};
\footnotemark[3]TU Delft,
Building 23,
Stevinweg 1 / PO-box 5048,
2628 CN Delft,
E-mail: \href{mailto:E.C.Slob@tudelft.nl}{E.C.Slob@tudelft.nl}.}


\author{%
Dieter Werthmüller\footnotemark[1], %
Kerry Key\footnotemark[2], and %
Evert C.\ Slob\footnotemark[3]%
}

\footer{}
\lefthead{Werthmüller et al.}
\righthead{Digital filter designing tool}

\maketitle

%%fakesection ===    ABSTRACT    ===
\begin{abstract} % 1-2 sentence(s) each
%
% (1) PRINCIPAL OBJECTIVES AND SCOPE OF THE WORK
  The open-source code \texttt{fdesign} makes it possible to design digital
  linear filters for the Hankel and Fourier transforms used in potential-,
  diffusive-, and wavefield modeling. Digital filters can be derived for any
  electromagnetic method, such as methods in the diffusive limits (DC, CSEM) as
  well as methods using higher frequency content (GPR, acoustic and elastic
  wavefields).
%
% (2) METHODOLOGY
  The direct matrix inversion method is used for the derivation of the filter
  values, and a brute-force inversion is carried out over the defined spacing
  and shifting values of the filter basis. Included or user-provided
  theoretical transform pairs are used for the inversion. Alternatively, one
  can provide 1D subsurface models that will be computed with a precise
  quadrature method using the EM modeler \texttt{empymod} to generate
  numerical transform pairs.
%
% (3) RESULTS
  The comparison of the presented 201\,pt filter with previously presented
  filters shows that it performs better for some standard CSEM cases. The
  derivation of longer 2001\,pt filter for a GPR example with 250\,MHz centre
  frequency proves that the filter method works also for wave phenomena, not
  only for diffusive EM fields.
%
% (4) CONCLUSIONS
  The presented algorithm provides a tool to create problem specific digital
  filters. Such purpose-built filters can be made shorter and can speed up
  consecutive  potential-, diffusive-, and wavefield inversions.
%
\end{abstract}

\section{Introduction}

Deva Prasad Ghosh proposed in his Ph.D. thesis \citep{PhD.70.Ghosh} a linear
filter method for the Hankel transform which revolutionized the computational
modeling of electromagnetic responses in the field of geophysical exploration.
If you use a code that calculates electromagnetic responses in the
wavenumber-frequency domain and transforms them to the space-frequency domain
chances are very high that it uses the \emph{digital linear filter} method
(DLF). \cite{GP.71.Ghosh} states that the idea is based on suggestions made
four decades earlier by \cite{PHY.33.Slichter} and \cite{GEO.40.Pekeris}, in
that ``\emph{the kernel function is dependent only on the layer parameters, and
  an expression relating it to the field measurements can be obtained by
mathematical processes.}'' However, until the introduction of DLF by Ghosh
these suggestions found no widespread use, probably not last because of the
missing computer power to calculate the filter coefficients. He further states
that credit goes to his Ph.D. supervisor Otto \cite{BK.68.Koefoed,
GP.70.Koefoed}, who retook the task of direct interpretational methods with the
introduction of raised kernel functions. DLF is, as such, an improvement of
that approach, providing a faster and simpler method.

% \subsection{Review}

The introduction of linear filters to electromagnetic geophysics initiated a
wealth of investigation, publications, and development of computer programs
that extended and improved the method. A few dozen articles have been published
in the 70s and 80s by Ghosh, Koefoed, Das, Anderson, and many others, the vast
majority of it in the journal \emph{Geophysical Prospecting}.
%  (see Figure~\ref{fig:authors})
Another common name of DLF is \emph{fast Hankel transform} (FHT), popular
because of the similarity of the name to the well known \emph{fast Fourier
transform} (FFT). The name was introduced, as far as we can see, by
\cite{GP.79.Johansen} in their article of the same name. However, the name FHT
can be misleading as it has \emph{Hankel} in the name, but the DLF can also be
applied to the Fourier transform (and other linear transforms).
% \begin{figure}[tbp]
%   \centering
%   \fbox{\includegraphics[width=10cm]{figures/author-figure}}
%     \caption{List by year of publications included in the review (year and
%       first three letters of principal author).}
%     \label{fig:authors}
% \end{figure}
Publications fall broadly into one or several of three categories: (1) new
methods (application of DLF to new measurement techniques); (2) filter
improvements (new filters or providing new or improved methods for the
determination of filter coefficients); and (3) computational tools. The
following is a brief review of the most relevant publications regarding DLF,
without claiming completeness.

\subsubsection{(1) New methods}

Ghosh used the method originally for the computation of type curves for
Schlumberger and Wenner resistivity soundings. He published the main results of
his thesis the following year in two publications: in \cite{GP.71.Ghosh} he
derives a resistivity model from given Schlumberger or Wenner sounding curves;
and in \cite{GP.71.Ghoshb} he provides filters for the inverse operation,
deriving resistivity sounding curves from a given resistivity model.  The
method was next applied to electromagnetic soundings with horizontal and
perpendicular coils \citep{GP.72.Koefoed}, to vertical coplanar coil systems
\citep{GP.73.Verma}, to dipoles and other two electrode systems
\citep{GP.74.Das, GP.74.Dasb, GP.80.Das, GEO.94.Sorensen}, and to vertical
dikes, hence vertical instead of horizontal layers \citep{GEO.75.Niwas}. Whilst
the first filters were very specific to one type of curves and one type of
transformation, various publications used the method to get one type curve from
another type curve \citep{GP.77.Kumar, GP.78.Kumar} or generalized the method
to be applicable to a wider set of problems \citep{EXG.80.Davis, GXP.81.Das,
GEO.84.Das, GP.84.ONeill}. Eventually, it passed from pure layered modeling to
primary-secondary formulations in 3D problems, where DLF is used to compute the
spatial Fourier-Hankel transforms in a horizontally layered background medium
and to compute transient responses from frequency domain computations
\citep{GJI.81.Das, GJI.82.Das, GEO.84.Anderson, GEO.86.Newman,
MGS.17.Kruglyakov}. Various authors delved into the theory of the method,
analyzing the oscillating behaviour of the filters and trying to estimate the
error of DLF \citep{GP.72.Koefoedb, GP.76.Koefoedb, GP.79.Johansen,
GP.90.Christensen}.

\subsubsection{(2) Filter improvements}

\cite{PhD.70.Ghosh} derived the filter coefficients in the spectral domain by
dividing the output spectrum by the input spectrum followed by an inverse
Fourier transform. Improvements to the determination of filter coefficients
were provided by \cite{EXG.75.ONeill, GEO.77.Nyman, GEO.82.Das}, or
specifically for the Fourier transform by \cite{GP.86.Nissen}. A direct
integration method was used by \cite{GP.76.Bichara, GP.78.Bernabini}.
\cite{GP.79.Koefoed} proposed a Wiener-Hopf least-squares method, which was
further improved by many authors \citep{GP.82.Guptasarma, GEO.1982.Murakami,
GP.97.Guptasarma}. \cite{GP.07.Kong} proposes a direct matrix inversion method
to solve the convolution equation, which requires only the input and output
sample values. To evaluate the filters he defines the criteria of a good filter
as one that recovers small or weak wavefields. This method was also used by
\cite{GEO.09.Key, GEO.12.Key}. Most authors publish filters for the Hankel
transform with $J_0$ and $J_1$ Bessel functions (or $J_{-1/2}$, $J_{1/2}$ if
applied to the Fourier sine/cosine transform), as all higher Bessel functions
can be rewritten to only use these two. \cite{GP.94.Mohsen} is one of the rare
cases which provides $J_2$ filter weights.

\subsubsection{(3) Codes}

The best known codes are likely the freely available ones by Anderson.
\cite{USGS.73.Anderson} extends the method to transient responses, applying DLF
not only to the Hankel transform, but also to the Fourier transform. A
transient signal can therefore be obtained be applying twice a digital filter
to the wavenumber-frequency domain calculation. In \cite{USGS.75.Anderson,
GEO.79.Anderson} he presents improved filters for both Fourier and Hankel
transforms, introducing measures to significantly speed-up the calculation,
such as the lagged convolution or using the same abscissae for $J_0$ and $J_1$.
\cite{TMS.82.Anderson} and the included 801\,pt filter became sort of the
industry standard, to which subsequent filters were compared. In
\cite{GEO.89.Anderson} he presents a hybrid solution, permitting to use either
DLF or quadrature and as such permits to compare the two. Other examples
include the codes by \cite{GP.75.Johansen}, an interactive system for
interpretation of resistivity soundings, and a tool to calculate filter
coefficients by \cite{GP.90.Christensen}. The latter is available upon request
and was used, for instance, in all the open-source modeling and inversion
routines of CSIRO in the Amira Project 223 \citep{ASEG.07.Raiche}. \newline

% \subsection{Outline}

All mentioned publications have in common that they were derived for direct
current methods (DC) or low frequency methods, such as time-domain shallow EM
methods (TEM), or controlled-source electromagnetics (CSEM), but not for high
frequency methods such as ground-penetrating radar (GPR). Generally it was even
thought that the filter method works only in the diffusive limit
\citep[e.g.,][]{GEO.15.Hunziker}. Also, there is no open-source filter
designing tool readily available, they are available as a described method
in articles or as code upon request by the author, if at all. The presented
algorithm \texttt{fdesign} tries to fill this gap by providing the tools to
design general or purpose-built filters using the direct matrix inversion
method. After a brief overview of the methodology, the theory, and the code we
show examples of its usage, presenting some new filters for both, CSEM and GPR
data. The algorithm and many more examples of its usage can be found on
\href{https://github.com/empymod/article-fdesign}{github.com/empymod/article-fdesign}.
The examples can be used as templates to design new filters.


\section{Methodology}

The algorithm \texttt{fdesign} is a filter designing tool using the direct
matrix inversion method as described by \cite{GP.07.Kong} and based on scripts
by \cite{GEO.12.Key}. The tool is an add-on to the electromagnetic modeler
\texttt{empymod} \citep{GEO.17.Werthmuller}, written in Python, and hosted on
GitHub, which should foster interaction and enable anyone to toy around,
improve, and extend it. It can be used to derive digital linear filters for the
Hankel and the Fourier transforms for potential-, diffusive-, and wavefields
(hence from DC to GPR). Theoretically, it can be used to derive linear filters
for any linear transform, as long as you feed it with a theoretical, or
accurately computed, transform pair of the transform. It permits to derive
filters with the help of theoretical transform pairs or, alternatively, with
the electromagnetic modeler \texttt{empymod}, using a quadrature method to
derive accurate curves used for the inversion.

The main points of the method and some differences between it and the methods
it is based upon are:
\begin{itemize}
  \item The algorithm computes many different filters for various spacing and
    shift values (brute-force).
  \item \cite{GP.07.Kong, GEO.12.Key} optimize a filter for the $J_0$ filter,
    and then use the obtained shift and spacing value to calculate the $J_1$
    values. The presented algorithm can optimize either $J_0$, $J_1$, or both
    in the optimization process. Five transform-pairs for each $J_0$ and $J_1$,
    and three transform pairs for each sine and cosine of often used functions
    are included in the routine, but any other transform pair can be provided
    as input.
  \item More complex or more specific models can be provided instead of
    theoretical transform pairs by using the modeler \texttt{empymod}.
  \item \cite{GP.07.Kong} defines a \emph{good} filter as one that recovers
    \emph{weak} diffusive EM fields. In the presented algorithm you can define
    a relative error level which defines up to what error an obtained result is
    good or not. The presented algorithm can also minimize the amplitude, but
    it has additionally another mode, where you maximise the abscissae $r$ of
    the right-hand-side of the transform pair instead of minimising the
    amplitude. The two modes yield the same result in the general, simple case
    of fast decaying transform pairs. However, if you use complex models for
    the design, and specifically high frequencies, then maximising $r$ will
    yield much better and more consistent results.
  \item The algorithm allows to run under-, equal-, and over-determined
    systems.
  \item the real or the imaginary part can be used for the inversion of complex
    transform pairs.
\end{itemize}

The quality of a filter depends heavily on the model chosen for comparison. An
obtained filter might be very good for one model, but not that good for another
one, which is no different for this designing tool. There is no way to estimate
the error of a result obtained with a certain filter if you apply it to any
other model, unless you calculate this other model with another method for
comparison. All results presented here should therefore be taken with a certain
care.

\subsection{Theory}

Most of the articles mentioned in the review have detailed derivations of the
digital filter method. In this article we focus on the algorithm, and summarize
the theory only very briefly by following \cite{GEO.12.Key} (equations
(2)-(6)). In electromagnetics we often have to evaluate integrals of the form
%
\begin{equation}
  F(r) = \int^\infty_0 f(l)K(l r)\mr{d}l \ ,
  \label{eq:HankelInt}
\end{equation}
%
where $l$ and $r$ denote left-hand-side and right-hand-side evaluation values,
and $K$ is the kernel function. In the specific case of the Hankel transform
$l$ corresponds to wavenumber, $r$ to offset, and $K$ to Bessel functions; in
the case of the Fourier transform $l$ corresponds to frequency, $r$ to time,
and $K$ to sine or cosine functions. In both cases it is an infinite integral
which numerical integration is very time-consuming because of the slow decay
of the kernel function and its oscillatory behaviour.

By substituting $r = e^x$ and $l = e^{-y}$ we get
\begin{equation}
  e^x F(e^x) = \int^\infty_{-\infty} f(e^{-y})K(e^{x-y})e^{x-y}\mr{d}y\ .
  \label{eq:filtint}
\end{equation}
This can be re-written as a convolution integral and be approximated for an
$N$-point filter by
\begin{equation}
  F(r) \approx \sum^N_{n=1} \frac{f(b_n/r) h_n}{r}\ ,
  \label{eq:filtapprox}
\end{equation}
where $h$ is the digital linear filter, and the logarithmically spaced filter
abscissae is a function of the spacing $\Delta$ and the shift $\delta$,
\begin{equation}
  b_n = \exp\left\{\Delta(-N/2+n) + \delta\right\} \ .
  \label{eq:base}
\end{equation}
From equation~\ref{eq:filtapprox} it can be seen that the filter method
requires $N$ evaluations at each $r$. So to calculate for instance the
frequency domain result for 100 offsets with a 201\,pt filter requires 20'100
evaluations in the wavenumber domain. This is why the DLF often uses
interpolation to minimise the required evaluations, either in the right-hand
side or in the left-hand side.

\subsection{Pseudo-code}
The main input variables are the filter length ($N$), the spacing ($\Delta$)
and shift ($\delta$) ranges over which to loop, and the transform pairs for
the inversion ($f_\mr{I}$) and the check of quality ($f_\mr{C}$). If
$f_\mr{C}$ is not provided, then $f_\mr{I}$ is used for both. There are
additional, optional input parameters, for instance to adjust how the rhs
abscissae $r = f(b, N)$ are calculated, where $b$ is the filter base.

The basic steps are as follows:
\begin{enumerate}
  \item Evaluate rhs of check-function $f_\mr{C}$:\newline
        $d_\mr{R} = f_\mr{C}.\mr{rhs}(r)$
  \item Loop over each value $\Delta_i$, $\delta_j$ (brute force):
    \begin{enumerate}
      \item Calculate filter base:\newline
        $b_n = \exp\left\{\Delta_i(-N/2+n) + \delta_j\right\}$
      \item Get required rhs ($r$) and lhs ($l$) evaluation points:\newline
        $r = f(b, N)$\newline
        $l = b/r$
      \item Invert for filter coefficients:\newline
        $J_{ij} = \mr{solve}(f_\mr{I}.\mr{lhs}(l), f_\mr{I}.\mr{rhs}(r))$
    \item Calculate numerically rhs of check-function $f_\mr{C}$ with current
      filter $J_{ij}$:\newline
      $d_\mr{F} = f_\mr{C}.\mr{lhs}(l)\cdot J_{ij}\ /\ r$
      \item Store minimum recovered amplitude or maximum $r$ where relative
        error is less than the provided error:\newline
        $\chi_{ij} = g\left[
        \mr{argmin}\left(|(d_\mr{F}-d_\mr{R})/d_\mr{R}| >
        \mr{error}\right) -1\right]\ $, where $g$ is either $d_\mr{R}$
        or $1/r$.
    \end{enumerate}
  \item Return filter coefficients which yield minimum amplitude or maximum $r$
    (a local minimization can be run to polish the brute-force result):\newline
    return $J\left[\mr{argmin}(\chi_{ij})\right] $\newline
\end{enumerate}

\subsection{Minimization criteria}
Figure~\ref{fig:InvCrit} shows the differences between the different
minimization approaches; in (a) for a conventional, fast decaying transform
pair; and in (b) for a more complex, high frequency layered model. The circles
show the minimum amplitude used in previous approaches. This criteria is not
ideal, as it is subject to some random fluctuations and also depends on the
choice of $r$. The squares show the minimum amplitude given a certain
acceptable error, and the diamonds show the maximum $r$ given a certain
acceptable error (the inversion is a minimization process, it therefore
minimizes $1/r$, not $r$). In simple cases the minimum amplitude given an
acceptable error and the maximum $r$ given an acceptable error will yield the
same result. However, in complex cases the maximum $r$ is more consistent and
therefore a much better criteria.
%
\plot*{InvCrit}{width=.9\textwidth}{(a) A regular rhs curve of a transform pair
  with a purely decaying function. Minimum amplitude or maximum $r$ yield the
  same result in this scenario. Using a relative error criteria is more stable
  than just the absolute minimum amplitude. (b) A 1D model for $f=80\,$MHz.
  Here, the maximum $r$ provides a better criteria for the inversion.}
%

It is important to note that independent of the inversion criteria the obtained
absolute value of it is only good to compare inversion results for the same
model. If you calculate filters with a different transform pair you might get
very different values, but this difference says nothing about the quality of
it. For instance, a certain transform pair might yield a lower minimum
amplitude than the other transform pair, but the resulting filter is worse if
both filters are compared to the same transform pair. This is due to the
characteristic of each transform pair.

\section{Numerical examples}

The numerical examples are focused on the Hankel transform, although
\texttt{fdesign} can be used to design digital linear filters for both Hankel
and Fourier transforms. As there is no difference in the procedure of the two,
this should be sufficient to demonstrate the algorithm.

\subsection{Design}
Figure \ref{fig:201} shows the solution spaces of four consecutive inversion
runs, where each run is a focus on a subsection of the previous run, indicated
by the red square.
%
\plot*{201}{width=\textwidth}{%
  Solution spaces of four consecutive inversion runs for a 201\,pt filter. Each
  consecutive run zooms into a portion of the previous solution space,
  indicated by the red square. The more in detail we obtain the solution space,
  the more random appears to be the distribution.}%
%
The filter length in this case is $N=201$, and the theoretical transform pairs
is (as given, e.g., in \cite{USGS.75.Anderson})
%
\begin{align}
  \int^\infty_0\,l \exp\left(-al^2\right) J_0(rl)\,dl &=
  \frac{\exp\left(\frac{-r^2}{4a}\right)}{2a}\ , \\
  \int^\infty_0\,l^2 \exp\left(-al^2\right) J_1(rl)\,dl &=
  \frac{r}{4a^2} \exp\left(-\frac{r^2}{4a}\right)\ ,
  \label{eq:j01}
\end{align}
%
where $a$ was set to 5. In the algorithm you can provide one pair for the inversion, and a different pair to get the minimum amplitude or the maximum $r$. In
this case the same transform pair was used for both the inversion and the check
of quality. The acceptable relative error was set to 1\,\%.
The rhs evaluation parameter for the inversion was set to $r_\mr{def} = (1, 1,
2)$, which means that an over-determined system was evaluated with $2N$
equations, where $r$ was logarithmically spaced from $\log_{10}(1/\max(b)) - 1$
to $\log_{10}(1/\min(b)) + 1$, where $b$ are the filter abscissae as given in
equation~\ref{eq:base}.

From the low resolution overview runs (a) and (b) it looks like this is a
standard, straight-forward minimization problem. However, from the more
detailed results in (c) and (d) it becomes obvious that it is a minimization
problem that has to be solved stochastically, as there are solutions with 2
orders of magnitude difference apparently randomly next to each other.

Figure \ref{fig:201b} shows in (a) the filter values for $J_0$ and $J_1$ of the
best 201 point filter obtained this way, and in (b) its right-hand-side
solution. The black dots indicate negative values, from which it can seen that
adjacent values are often alternating between positive and negative
contributions.
%
\plot*{201b}{width=.9\textwidth}{(a) Filter values of the best obtained 201\,pt
filter with the corresponding check of quality in (b). Black points indicate
negative values, which shows that adjacent values have often opposite signs.}
%

Designing filters is to a large extent trial and error. All input variables
influence the outcome, and often you will come across a good filter by shear
luck of choosing the right starting parameters. Each of the input parameters
has an effect to the outcome. It depends a lot on the transform pairs
$f_\mr{I;C}$, and functions that decay rapidly are generally better, as noted
by earlier authors \citep[e.g.][]{USGS.75.Anderson}. It also depends on the
filter length, and obviously heavily on the spacing and shift values, as this
is what we invert for. Another important point is how you define the right hand
evaluation points of the inversion ($r_\mr{def}$). Evaluating corresponding
transform pairs separately or jointly also leads to different filter
coefficients ($J_0$, $J_1$, or $J_0$ \& $J_1$, or equally sine, cosine, or sine
\& cosine); if the real or the imaginary part is used when complex transform
pairs or \texttt{empymod} is used; and if it is inverted for the minimum
amplitude or for the maximum $r$.

\subsection{CSEM}
In this section we compare the 201\,pt filter derived in the previous section
to CSEM models used in \cite{GP.07.Kong}, \cite{GEO.12.Key}, and a land case.

Figures \ref{fig:kongf5} compares the derived 201\,pt filter with the two
half-spaces case used by \cite{GP.07.Kong} in his figure 5. The model consists
of a water layer with $\rho_\mr{w} = 0.3125\,\Omega\,$m of infinite thickness
above a subsurface half-space with $\rho = 1\,\Omega\,$m. The signal of an
$x$-directed electric source 50\,m above the interface is measured at an
$x$-directed electric receiver at the interface, frequency is $f=1\,$Hz.
%
\plot*{kongf5}{width=.9\textwidth}{(a) Results of different filters and QWE for
  the model of figure 5 from \cite{GP.07.Kong} with the relative errors shown
  in (b), using the QWE result. The relative error is meaningless from about
  15.5\,km onwards, as QWE failed as well for these very low amplitudes.}
%
In (a) it can be seen that the new 201\,pt filter is able to recover smaller
amplitudes than the 241\,pt filter from \cite{GP.07.Kong}, the 201\,pt filter
from \cite{GEO.12.Key}, and the 801\,pt filter from \cite{TMS.82.Anderson}
(Wer201, Kong241, Key201, and And801, respectively). It behaves equally well as
the quadrature with extrapolation (QWE), for which we used a 51\,pt quadrature
with relative and absolute tolerance of 1e-12 and 1e-30, respectively. The
relative error is shown in (b), where the QWE result was taken as \emph{truth}.
For offsets greater than roughly 15.5\,km the relative error becomes
meaningless, as the QWE fails itself; this part is greyed out in the figure.

Figure \ref{fig:keyf5} (a) is the canonical CSEM model as used in
\cite{GEO.12.Key} in his figure 5: Water depth of 2\,km with
$\rho_\mr{w}=0.303\,\Omega\,$m, a background resistivity of
$\rho_\mr{b}=1\,\Omega\,$m, in which a target is embedded of
$\rho_\mr{t}=100\,\Omega\,$m at 1000\,m below the seafloor, 100\,m thick.
Source depth is 1990\,m, receivers are on the seafloor.
%
\plot*{keyf5}{width=.9\textwidth}{(a) The canonical marine CSEM model of
  \cite{GEO.12.Key} and (b) a land model with source and receiver at the
  surface. The new filter has generally relative errors which are several
  magnitudes lower than the other filters.}
%
Figure \ref{fig:keyf5} (b) is a land case, with a background resistivity of
$\rho_\mr{b}=10\,\Omega\,$m, in which a target is embedded of
$\rho_\mr{t}=500\,\Omega\,$m at 1000\,m below the surface, 100\,m thick. Source
depth is 0.5\,m, receiver depth is 0.8\,m. In both cases the new filter
\emph{Wer201} has generally a relative error which is magnitudes lower than the
other filters.

It is very important to note again that other scenarios might yield very
different error plots. Although the new 201\,pt filter proves to be very
accurate for these three models, it might not be the best filter in other
cases.

\subsection{Ground-penetrating radar}

\rmk{TODO: Add second GPR result; layers swapped}

Figure~\ref{fig:GPR} shows the GPR example as calculated in
\cite{GEO.15.Hunziker, GEO.17.Werthmuller}; the model parameters are given in
subplot (a). The filter used for this example for the Hankel transform is a
2001\,pt filter, derived with the fullspace solution with $f=500\,$MHz for the
inversion and the check of quality. Vertical source and receiver separation is
1\,m, resistivity $\rho=200\,\Omega\,$m, $\varepsilon_\mr{r}$=10,
$\mu_\mr{r}=1$. For the Fourier transform a 4096\,pt FFT was used with
regularly spaced frequencies from 0.5\,MHz to 850\,MHz and then zero-padded up
to 2048\,MHz for both the calculation with the adaptive quadrature (QUA) and
with the DLF. The frequency result is convolved with a Ricker wavelet with a
center frequency of 250\,MHz, and a gain function ($1 + |t^3|, t$ in ns) is
applied. The calculation with the digital filter took under 9 minutes and is
therefore roughly 80 times faster then the adaptive quadrature calculation
which took roughly 11 hours and 27 minutes. Calculating the same model with QWE
took 7 hours and 20 minutes. However, QWE uses in roughly 1/3 of the
calculation internally the adaptive quadrature in this example, see
\cite{GEO.17.Werthmuller}. (Note that DLF was run in parallel using 4 threads
at once, taking effectively only 2 minutes and 10 seconds to calculate.
%EMmod \citep{GEO.15.Hunziker} took for the same model about 18 hours 55
%minutes, using a 61\,pt Gauss-Kronrod integration routine.
The lagged convolution version of DLF and the splined version of QWE were used
in this comparison.)
%
\plot*{GPR}{width=\textwidth}{GPR example for the model given in (a) using
  \texttt{empymod} with (b) the adaptive quadrature (QAGSE from the Fortran
  QUADPACK library) or with (c) a 2001\,pt digital linear filter (b). The
  quadrature took about 11\,h 27\,min to calculate, whereas the filter took
  2\,min 10\,s.}
%

Figure~\ref{fig:ftGPR} shows in (a)-(c) the real part of the frequency domain
results and in (d)-(f) the time domain results for offsets of 0.2\,m, 2.0\,m,
and 3.0\,m.
%
\plot*{ftGPR}{width=\textwidth}{The real-part frequency-domain and the
time-domain responses for offsets of (a) 0.2\,m, (b) 2.0\,m, and (c) 3.0\,m.}
%
These examples show clearly that the filter method can indeed be applied to
high frequency EM modeling and therefore wave phenomena. We are convinced that
with further tests and analysis much better filters could be achieved, and
various concepts could be checked. One approach is to derive a filter for each
frequency band, say one for 1\,MHz--10\,MHz, one for 10\,MHz--100\,MHz, and one
for 100\,MHz--1\,GHz. Another approach could be to derive distinct filters for
$J_0$ and $J_1$, with different spacing and shift values. The first idea would
triple the calculation cost, the second idea double them; however, they both
would still be very fast compared to standard quadrature methods.

\section{Conclusions}

The presented, free and open-source algorithm \texttt{fdesign} can be used to
design digital linear filters for the Hankel and Fourier transforms (and more
generally for any linear transform) using either analytical transform pairs or
1D subsurface models together with the EM-modeler \texttt{empymod}.
The code is available from GitHub in the add-on repository of \texttt{empymod},
called \texttt{empyscripts}, from version 0.1.2 onwards.

The presented 201\,pt filter achieves more precise results in the three
presented CSEM cases than other filters, and is included in \texttt{empymod}
from version 1.4.5 onwards. However, as with any digital filter, the quality
depends heavily on the model, and this new filter might or might not behave
that well for other models.

The shown GPR result shows that the digital linear filter method can also be
used for wave phenomena, not only for the diffusive approximation limit of low
frequency EM modeling.

We hope the presented algorithm is useful for at least 3 scenarios:
\begin{enumerate}
  \item Provide a fast method to design problem-specific filters. For bigger
    inversion projects (such as, for instance, massive stochastic CSEM
    inversions) a purpose designed, short filter might save much more time than
    it costs to design it. This could even be integrated into inversion codes,
    as an optional pre-inversion step.
  \item Extend the filter method to new areas, namely higher frequencies.
  \item Raise a new interest for digital linear filters in geophysics, by
    making it very easy for anyone to play around, create their own filters,
    and get a better understanding of it. We are sure there are many great
    filters to be discovered.
\end{enumerate}

\section{Acknowledgment}
D.W. would like to thank his colleague Oleg Gurevich Titov, who introduced him
to linear digital filters in the first place.
% This project would not have been successful without the fruitful discussions
% with Evert Slob and Kerry Key. Many thanks as well to Kerry Key for providing
% me with his Matlab scripts, which served as a starting point and initiated
% this project.

% REFERENCES
\bibliographystyle{seg}
\bibliography{\string~/.dotfiles/LaTeX/dtrRef}

\end{document}
